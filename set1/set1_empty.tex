\newif\ifshowsolutions
\showsolutionstrue

\input{../preamble}

\chead{
  {\vbox{
      \vspace{2mm}
      \large
      Machine Learning \& Data Mining \hfill
      Caltech CS/CNS/EE 155 \hfill \\[1pt]
      Set 1\hfill
      January $6^{th}$, 2023 (LATE: 4 hours) \\
    }
  }
}

\begin{document}
\pagestyle{fancy}



%%%%%%%%%%%%%%%%%%%%%%%%%%%%%%
% POLICIES
%%%%%%%%%%%%%%%%%%%%%%%%%%%%%%

\section*{Policies}
\begin{itemize}
	\item Due 9 PM PST, January $13^\text{th}$ on Gradescope. 
	\item You are free to collaborate on all of the problems, subject to the collaboration policy stated in the syllabus.
	\item If you have trouble with this homework, it may be an indication that you should drop the class.
	\item In this course, we will be using Google Colab for code submissions. You will need a Google account.
\end{itemize}

\section*{Submission Instructions}

\begin{itemize}
	\item Submit your report as a single .pdf file to Gradescope (entry code K3RPGE), under "Set 1 Report". 
	\item In the report, \textbf{include any images generated by your code} along with your answers to the questions.
	\item Submit your code by \textbf{sharing a link in your report} to your Google Colab notebook for each problem (see naming instructions below). Make sure to set sharing permissions to at least "Anyone with the link can view". \textbf{Links that can not be run by TAs will not be counted as turned in.} Check your links in an incognito window before submitting to be sure. 
	\item For instructions specifically pertaining to the Gradescope submission process, see \url{https://www.gradescope.com/get_started#student-submission}.
\end{itemize}


\section*{Google Colab Instructions}

For each notebook, you need to save a copy to your drive.

\begin{enumerate}
	\item Open the github preview of the notebook, and click the icon to open the colab preview.
	\item On the colab preview, go to File $\rightarrow$ Save a copy in Drive.
	\item Edit your file name to “lastname_firstname_originaltitle”, e.g.”yue_yisong_3_notebook_part1.ipynb”
\end{enumerate}

%%%%%%%%%%%%%%%%%%%%%%%%%%%%%%
% PROBLEM 1
%%%%%%%%%%%%%%%%%%%%%%%%%%%%%%

\newpage
\section{Basics [16 Points]}
\materials{lecture 1}

Answer each of the following problems with 1-2 short sentences.

\begin{problem}[2]
  What is a hypothesis set?
\end{problem}
\begin{solution}
A hypothesis set is a collection of functions, that are built off a certain model, that classify a given input set $\mathcal{X}$ to output set $\mathcal{Y}$.
\end{solution}

\begin{problem}[2]
  What is the hypothesis set of a linear model?
\end{problem}
\begin{solution}
  The perceptron. In other words, a line that classifies vectors in $\mathcal{X}$ depending on which side of the line the vector is on. It is also described by the following:
  $$
  w^Tx+b
  $$
\end{solution}

\begin{problem}[2]
  What is overfitting?
\end{problem}
\begin{solution}
  Overfitting occurs when, during training, a hypothesis is chosen that increases the in-sample error but decreases the out-of-sample error. Intuitively, this happens when characteristics of the specific data set (which really shouldn't be learned) are incorporated into the hypothesis. This can cause significant error when evaluating out-of-sample data.
\end{solution}

\begin{problem}[2]
  What are two ways to prevent overfitting?
\end{problem}
\begin{solution}
  We can perform training validation. In other words, we choose a subset of $\mathcal{X}$ to be our validation set ($\mathcal{X}_V$), while the disjoint of $\mathcal{X}_V$ is our training set, $\mathcal{X}_T$. We then train on $\mathcal{X_T}$ and stop when the accuracy of predicting $\mathcal{X}_V$ decreases. We can also reduce overfitting using a less complex model (i.e. regularization). Furthermore, we can also increasing the amount of training data.
\end{solution}
\newpage
\begin{problem}[2]
  What are training data and test data, and how are they used differently? Why should you never change your model based on information from test data?
\end{problem}
\begin{solution}
  Training data is used to select the 'best' hypothesis we can find. Test data is used to evaluate that hypothesis (namely, to see how good it is at predicting the test data). We don't want to change our model based on the test data because that could potentially lead to overfitting. Furthermore, test data can be skewed, which would cause the model to perform worse overall on other test data. In other words \textbf{we do not expect to find the entire distribution} in the test data.
\end{solution}

\begin{problem}[2]
  What are the two assumptions we make about how our dataset is sampled?
\end{problem}
\begin{solution}
    We note that there will likely always be noise and that the entire distribution will likely not be described by the dataset. Furthermore, the dataset is sampled identically and independently from the population.
\end{solution}

\begin{problem}[2]
  Consider the machine learning problem of deciding whether or not an email is spam. What could $X$, the input space, be? What could $Y$, the output space, be?
\end{problem}
\begin{solution}
  $X$ could be the some feature input set that descibes a given email. For example, we could sample from a given language "Bag of Words" that describes certain words in the email. We can then label $Y$ as the output set, or whether an email is spam or not (or $1$ and $-1$ respectively). 
\end{solution}

\begin{problem}[2]
  What is the $k$-fold cross-validation procedure?
\end{problem}
\begin{solution}
  This problem relates to grouping $n$ vectors into $k$ groups, where $n >> k$. These groups are then used to train $k-1$ models, while the remaining model is used as an "out-of-sample" testing set to validate the data. This is commonly used in regression problems or radial basis functions.


\end{solution}



%%%%%%%%%%%%%%%%%%%%%%%%%%%%%%
% PROBLEM 2
%%%%%%%%%%%%%%%%%%%%%%%%%%%%%%

\newpage
\section{Bias-Variance Tradeoff [34 Points]}
\materials{lecture 1}

\begin{problem}[5]
  Derive the bias-variance decomposition for the squared error loss function. That is, show that for a model $f_S$ trained on a dataset $S$ to predict a target $y(x)$ for each $x$,
  \begin{align*}
    \E_S \left[E_\text{out}\left(f_S\right)\right] = \E_x[\Bias(x) + \Var(x)]
  \end{align*}
  given the following definitions:
  \begin{align*}
    F(x) &= \E_S\left[f_S(x) \right] \\
    E_\text{out}(f_S) &= \E_x\left[\left(f_S(x) - y(x)\right)^2\right] \\
    \Bias(x) &= (F(x) - y(x))^2 \\
    \Var(x) &= \E_S\left[(f_S(x)-F(x))^2\right]
  \end{align*}
\end{problem}

\begin{solution}
% $$
% \E_x[\Bias(x) + \Var(x)]
% $$$$
% = \E_x\left[(F(x) - y(x))^2 +  \E_S\left[(f_S(x)-F(x))^2\right]\right]
% $$
% See that $\E_S$ simply finds the expected value over
$$
\E_S \left[E_\text{out}\left(f_S\right)\right]
$$$$
= \E_S \left[  \E_x\left[\left(f_S(x) - y(x)\right)^2\right] \right]
$$$$
= \E_x \left[ \E_S\left[\left(f_S(x) - y(x)\right)^2\right] \right]
$$$$
= \E_x \left[ \E_S\left[f_S(x)^2 - 2f_S(x)y(x) + y(x)^2\right] \right]
$$$$
= \E_x \left[F(x)^2 -2F(x)y(x) + y(x)^2 \right]
$$$$
= \E_x \left[(F(x) - y(x))^2 \right]
$$$$
= \E_x \left[\Bias(x) + \E_S\left[(f_S(x)-F(x))^2\right] \right]
$$$$
= \E_x[\Bias(x) + \Var(x)]
$$
\end{solution}

In the following problems you will explore the bias-variance tradeoff by producing learning curves for polynomial regression models.

A \emph{learning curve} for a model is a plot showing both the training error and the cross-validation error as a function of the number of points in the training set. These plots provide valuable information regarding the bias and variance of a model and can help determine whether a model is over-- or under--fitting.

\emph{Polynomial regression} is a type of regression that models the target $y$ as a degree--$d$ polynomial function of the input $x$. (The modeler chooses $d$.)  You don't need to know how it works for this problem, just know that it produces a polynomial that attempts to fit the data.

\begin{problem}[14]
    Use the provided \texttt{2_notebook.ipynb} Jupyter notebook to enter your code for this question. This notebook contains examples of using NumPy's polyfit and polyval methods, and scikit-learn's KFold method; you may find it helpful to read through and run this example code prior to continuing with this problem. Additionally, you may find it helpful to look at the documentation for scikit-learn's learning_curve method for some guidance.

The dataset \texttt{bv_data.csv} is provided and has a header denoting which columns correspond to which values. Using this dataset, plot learning curves for 1st--, 2nd--, 6th--, and 12th--degree polynomial regression (4 separate plots) by following these steps for each degree $d \in \{1, 2, 6, 12\}$:

  \begin{enumerate}
    \item For each $N \in \{20, 25, 30, 35, \cdots, 100\}$:
    \begin{enumerate}[i.]
      \item Perform 5-fold cross-validation on the first $N$ points in the dataset (setting aside the other points), computing the both the training and validation error for each fold. 
      \begin{itemize}
        \item Use the mean squared error loss as the error function.
        \item Use NumPy's polyfit method to perform the degree--$d$ polynomial regression and NumPy's polyval method to help compute the errors.  (See the example code and \href{https://docs.scipy.org/doc/NumPy/reference/routines.polynomials.poly1d.html}{NumPy documentation} for details.)
        \item When partitioning your data into folds, although in practice you should randomize your partitions, for the purposes of this set, simply divide the data into $K$ contiguous blocks.
      \end{itemize}
      \item Compute the average of the training and validation errors from the 5 folds.
    \end{enumerate}
    \item Create a learning curve by plotting both the average training and validation error as functions of $N$. \textit{Hint: Have same y-axis scale for all degrees $d$.}
  \end{enumerate}

\end{problem}
\begin{solution}
\href{https://colab.research.google.com/drive/1vB_mlWWWBEmudPGr49NszSYmXRUAdfZg?usp=sharing}{CODE HERE} \\
\begin{center}
    \includegraphics[width=8cm]{set1/images/output.png}
\end{center}
  \end{solution}

\begin{problem}[3]
  Based on the learning curves, which polynomial regression model (i.e. which degree polynomial) has the highest bias? How can you tell?
\end{problem}
\begin{solution}
Degree 1 has the highest bias since the training error is the largest (and plateaus) and also because it is the simplest model.
\end{solution}

\begin{problem}[3]
  Which model has the highest variance? How can you tell?
\end{problem}
\begin{solution}
  The model with the highest variance is degree 12. We know this because the out of sample error is quite high and because the model is very complex.
\end{solution}

\begin{problem}[3]
  What does the learning curve of the quadratic model tell you about how much the model will improve if we had additional training points?
\end{problem}
\begin{solution}
We note that as we add more points, the in sample and out of sample error both plateau. Thus, performance will improve only marginally with additional points.
\end{solution}

\begin{problem}[3]
  Why is training error generally lower than validation error?
\end{problem}
\begin{solution}
  This is because the model is fitted to the training data. In contrast, it has never 'seen' the validation data. Thus, validation error is higher as it is out-of-sample.
\end{solution}

\begin{problem}[3]
  Based on the learning curves, which model would you expect to perform best on some unseen data drawn from the same distribution as the training data, and why?
\end{problem}
\begin{solution}
  Likely degree 6 since it has the lowest validation error, which is equivalent to "unseen" data.
\end{solution}




%%%%%%%%%%%%%%%%%%%%%%%%%%%%%%
% PROBLEM 3
%%%%%%%%%%%%%%%%%%%%%%%%%%%%%%

\newpage
\section{Stochastic Gradient Descent [34 Points]}
\materials{lecture 2}

Stochastic gradient descent (SGD) is an important optimization method in machine learning, used everywhere from logistic regression to training neural networks. In this problem, you will be asked to analyze gradient descent and implement SGD for linear regression using the squared loss function. Then, you will analyze how several parameters affect the learning process.

\begin{problem}[3]
    To verify the convergence of our gradient descent algorithm, consider the task of minimizing a function $f$ (assume that $f$ is continuously differentiable). Using Taylor's theorem, show that if $x'$ is a local minimum of $f$, then $\nabla f(x') = 0$. 
\end{problem}
\begin{hint}
  First-order Taylor expansion gives that for some $x, h \in \mathbb{R}^n$, there exists $c \in (0, 1)$ such that $f(x + h) = f(x) + \nabla f(x + c\cdot h)^T h$.
  
\end{hint}
\begin{solution}
Assume, to the contrary that $\nabla f(x') \neq 0$. Allow $h = \varepsilon \nabla f(x'), \varepsilon > 0$. Then by the first order Taylor series we have
$$
f(x' - \varepsilon \nabla f(x')) = f(x') - \varepsilon \nabla f(x' - c \varepsilon \nabla f(x'))^T\nabla f(x')
$$
since $\nabla f$ is continuous and from Taylors theorem, we can also write that
$$
\nabla f(x' - c \varepsilon \nabla f(x'))^T \geq \frac{1}{2} || \nabla f(x') ||
$$
which means that $\nabla f(x' - c \varepsilon \nabla f(x'))^T$ is \textbf{not} negatively valued. Notice that if $\nabla f(x')$ were positive (nonzero), then we would have some $x''$ with$f(x'') < f(x')$, which is impossible. Thus (contradiction) we can only write that $\nabla f(x') = 0$.
\end{solution}
\newpage
Linear regression learns a model of the form:
\begin{align*}
  f(x_1, x_2, \cdots, x_d) = \left(\sum_{i=1}^d w_i x_i\right) + b
\end{align*}

\begin{problem}[1]
  We can make our algebra and coding simpler by writing $f(x_1, x_2, \cdots, x_d) = \mathbf{w}^T\mathbf{x}$ for vectors $\mathbf{w}$ and $\mathbf{x}$.  But at first glance, this formulation seems to be missing the bias term $b$ from the equation above.  How should we define $\mathbf{x}$ and $\mathbf{w}$ such that the model includes the bias term?
\end{problem}
\begin{hint}
  Include an additional element in $\mathbf{w}$ and $\mathbf{x}$.
\end{hint}
\begin{solution}
  We can simply concatenate an extra 1 to the beginning of $\textbf{x}$. This will cause our weight vector $\textbf{w}$ to be of dimension $N + 1$, where $N = \dim(x)$. In other words, we will have that
  $$
  w_0 \cdot 1 + \sum_{i=1}^N w_i \cdot x_i
  $$
  where $\textbf{w} = \langle w_0, w_1, ..., w_N \rangle$ and symmetrically for $\textbf{x}$.
\end{solution}

Linear regression learns a model by minimizing the squared loss function $L$, which is the sum across all training data $\{(\mathbf{x}_1, y_1),\cdots,(\mathbf{x}_N, y_N)\}$ of the squared difference between actual and predicted output values:
\[L(f) = \sum_{i=1}^N (y_i - \mathbf{w}^T\mathbf{x}_i)^2\]

\begin{problem}[2]
  Both GD and SGD uses the gradient of the loss function to make incremental adjustments to the weight vector $\mathbf{w}$. Derive the gradient of the squared loss function with respect to $\mathbf{w}$ for linear regression. Explain the difference in computational complexity in 1 update of the weight vector between GD and SGD. 
\end{problem}
\begin{solution}
  $$
  \partial_w L(w,b) = \partial \sum_{i = 1}^N L(y_i,f(x | w,b)
  $$$$
  = \sum_{i = 1}^N \partial_w L(y_i, f(x_i | w, b))
  $$$$
  = \sum -2(y_i - f(x_i | w, b))\partial_w f(x_i | w,b))
  $$$$
  = \sum_{i=1}^N -2(y_i - (w^Tx - b))x_i
  $$

  Gradient descent requires iterating through the entire dataset, which is O(N) time complexity. By contrast, since we assume, in expectation, batches will perform the same over multiple epochs, we may use stochastic gradient descent. Here randomly chosen batches (which are subsets of the full dataset) are chosen. This is technically still linear time, but with a sufficient number of batches SGD is much faster (due to a smaller number of points to iterate over per update).
\end{solution}

The following few problems ask you to work with the first of two provided Jupyter notebooks for this problem, \texttt{3_notebook_part1.ipynb}, which includes tools for gradient descent visualization. This notebook utilizes the files \texttt{sgd_helper.py} and \texttt{multiopt.mp4}, but you should not need to modify either of these files. 

For your implementation of problems D-F, \textbf{do not} consider the bias term.

\begin{problem}[6]
  Implement the \texttt{loss}, \texttt{gradient}, and \texttt{SGD} functions, defined in the notebook, to perform SGD, using the guidelines below:

  \begin{itemize}
    \item Use a squared loss function.
    \item Terminate the SGD process after a specified number of epochs, where each epoch performs one SGD iteration for each point in the dataset.
    \item It is recommended, but not required, that you shuffle the order of the points before each epoch such that you go through the points in a random order. You can use \texttt{numpy.random.permutation}.
    \item Measure the loss after each epoch. Your \texttt{SGD} function should output a vector with the loss after each epoch, and a matrix of the weights after each epoch (one row per epoch). Note that the weights from all epochs are stored in order to run subsequent visualization code to illustrate SGD.
  \end{itemize}
\end{problem}
\begin{solution}
\href{https://colab.research.google.com/drive/1GFc23TqByol_4OWS_tyVTa3n8eoYXFvn?usp=sharing}{CODE HERE}
\end{solution}

\begin{problem}[2]
  Run the visualization code in the notebook corresponding to problem D. How does the convergence behavior of SGD change as the starting point varies? How does this differ between datasets 1 and 2? Please answer in 2-3 sentences.
\end{problem}
\begin{solution}
As we change the starting point, SGD takes longer to converge for points further away from the minimum. However, between datasets 1 and 2 there isn't a large difference in convergence time.
\end{solution}

\begin{problem}[6]
  Run the visualization code in the notebook corresponding to problem E. One of the cells---titled "Plotting SGD Convergence"---must be filled in as follows. Perform SGD on dataset 1 for each of the learning rates $\eta \in \{10^{-6}, 5 \cdot 10^{-6}, 10^{-5}, 3 \cdot 10^{-5}, 10^{-4}\}$. On a single plot, show the training error vs. number of epochs trained for each of these values of $\eta$. What happens as $\eta$ changes?
\end{problem}

\begin{solution}
It takes longer to converge for smaller $\eta$. The largest $\eta$ are the ones on top (on the right of the graph) and they go down to the smallest $\eta$. (Could not figure out how to get legend).

\begin{center}
    \includegraphics[width=10cm]{set1/images/eta1.png}
\end{center}
\end{solution}


The following problems consider SGD with the larger, higher-dimensional dataset, \texttt{sgd_data.csv}. The file has a header denoting which columns correspond to which values. For these problems, use the Jupyter notebook \texttt{3_notebook_part2.ipynb}.

For your implementation of problems G-I, \textbf{do} consider the bias term using your answer to problem A.

\begin{problem}[6]
  Use your SGD code with the given dataset, and report your final weights. Follow the guidelines below for your implementation:

  \begin{itemize}
    \item Use $\eta = e^{-15}$ as the step size.  
    \item Use $\mathbf{w} = [0.001, 0.001, 0.001, 0.001]$ as the initial weight vector and $b = 0.001$ as the initial bias.
    \item Use at least 800 epochs.
    \item You should incorporate the bias term in your implementation of SGD and do so in the vector style of problem A.
    \item Note that for these problems, it is no longer necessary for the \texttt{SGD} function to store the weights after all epochs; you may change your code to only return the final weights.
  \end{itemize}
  %$\epsilon$ here is a measure of how much change in error there is compared to the initial error in the epoch. Calculate the change in error every epoch and compare it to the change in error from the first epoch. If new change/initial change is less than $\epsilon$, stop the training. $\eta$ is the factor by which you multiply the gradient in each step of the descent, and $\mathbf{w}$ is the initial weight vector.
\end{problem}
\begin{solution}
\href{https://colab.research.google.com/drive/1tpWQzEpdqmPHiF7JQ3pcArDQjDq3R2UX?usp=sharing}{CODE HERE}
 $$w = -5.94211251, 3.94389978, -11.7238423, 8.78567496
 $$$$ 
 b = -0.22718924
 $$
\end{solution}

\begin{problem}[2]
  Perform SGD as in the previous problem for each learning rate $\eta$ in \[\{e^{-10}, e^{-11}, e^{-12}, e^{-13}, e^{-14}, e^{-15}\},\] and calculate the training error at the beginning of each epoch during training.  On a single plot, show training error vs. number of epochs trained for each of these values of $\eta$. Explain what is happening.
\end{problem}
\begin{solution}
  It takes longer to converge for smaller $\eta$.
  \begin{center}
    \includegraphics[width=10cm]{set1/images/eta2.png}
\end{center}
\end{solution}


\begin{problem}[2]
  The closed form solution for linear regression with least squares is \[\mathbf{w} = \left(\sum_{i=1}^N \mathbf{x_i}\mathbf{x_i}^T\right)^{-1}\left(\sum_{i=1}^N \mathbf{x_i}y_i\right).\]  Compute this analytical solution.  Does the result match up with what you got from SGD?
\end{problem}
\begin{solution}
 Yes, for the SGD we have that 
 $$w = -5.94211251, 3.94389978, -11.7238423, 8.78567496
 $$$$ 
 b = -0.22718924
 $$
 And for closed form we have
 $$w = -5.99157048, 4.01509955, -11.93325972, 8.99061096
 $$$$ 
 b = -0.31644251
 $$
 Which is pretty close.
\end{solution}

Answer the remaining questions in 1-2 short sentences.

\begin{problem}[2]
  Is there any reason to use SGD when a closed form solution exists?
\end{problem}
\begin{solution}
  Since SGD works on subsets of data (whereas the closed form solution uses the entire dataset), SGD can be computationally faster.
\end{solution}

\begin{problem}[2]
  Based on the SGD convergence plots that you generated earlier, describe a stopping condition that is more sophisticated than a pre-defined number of epochs.
\end{problem}
\begin{solution}
  We may observe the euclidean distance between iterative weight vectors and stop when it reaches a certain lower threshold.
\end{solution}


%%%%%%%%%%%%%%%%%%%%%%%%%%%%%%
% PROBLEM 4
%%%%%%%%%%%%%%%%%%%%%%%%%%%%%%

\newpage
\section{The Perceptron [16 Points]}
\materials{lecture 2}

The perceptron is a simple linear model used for binary classification. For an input vector $\mathbf{x} \in \mathbb{R}^d$, weights $\mathbf{w} \in \mathbb{R}^d$, and bias $b \in \mathbb{R}$, a perceptron $f: \mathbb{R}^d \rightarrow \{-1,1\}$ takes the form
\begin{align*}
  f(\mathbf{x}) = \operatorname{sign}\left(\left(\sum_{i=1}^d w_i x_i\right) + b \right)
\end{align*}

The weights and bias of a perceptron can be thought of as defining a hyperplane that divides $\mathbb{R}^d$ such that each side represents an output class. For example, for a two-dimensional dataset, a perceptron could be drawn as a line that separates all points of class $+1$ from all points of class $-1$.

The PLA (or the Perceptron Learning Algorithm) is a simple method of training a perceptron. First, an initial guess is made for the weight vector $\mathbf{w}$. Then, one misclassified point is chosen arbitrarily and the $\mathbf{w}$ vector is updated by
\begin{align*}
  \mathbf{w}_{t+1} &= \mathbf{w}_t + y(t)\mathbf{x}(t) \\
  b_{t + 1} &= b_t + y(t),
\end{align*}

where $\mathbf{x}(t)$ and $y(t)$ correspond to the misclassified point selected at the $t^\text{th}$ iteration.
This process continues until all points are classified correctly.

The following few problems ask you to work with the provided Jupyter notebook for this problem, titled \texttt{4_notebook.ipynb}. This notebook utilizes the file \texttt{perceptron_helper.py}, but you should not need to modify this file.

\begin{problem}[8]
  The graph below shows an example 2D dataset. The $+$ points are in the $+1$ class and the $\circ$ point is in the $-1$ class. 

  \begin{figure}[H]
    \centering
    \includegraphics[width=0.4\textwidth]{images/perceptron.png}
    \caption{The green $+$ are positive and the red $\circ$ is negative}
    \label{fig:figure1}
  \end{figure}
  
 Implement the \texttt{update_perceptron} and \texttt{run_perceptron} methods in the notebook, and perform the perceptron algorithm with initial weights $w_1 = 0, w_2 = 1, b = 0$.

  Give your solution in the form a table showing the weights and bias at each timestep and the misclassified point $([x_1,x_2],y)$ that is chosen for the next iteration's update. You can iterate through the three points in any order. Your code should output the values in the table below; cross-check your answer with the table to confirm that your perceptron code is operating correctly.

  \begin{table}[H]
    \centering

    \begin{tabular}{l|lll|ll|l}
    \hline

    \hline
    $t$ & $b$ & $w_1$ & $w_2$ & $x_1$ & $x_2$ & $y$ \\
    \hline
      0  &  0 & 0 & 1  & 1 & -2 & +1\\
      1  &  1 & 1 & -1 & 0 & 3 & +1\\
      2  &  2 & 1 & 2 & 1 & -2 & +1\\
      3  &  3 & 2 & 0 \\
    \hline
    \end{tabular}
  \end{table}
  
  Include in your report both: the table that your code outputs, as well as the plots showing the perceptron's classifier at each step (see notebook for more detail).
  
  
\end{problem}
\begin{solution}
\href{https://colab.research.google.com/drive/1Gq4s0ZueixyhST93Yo0EFipu_uvUbHx9?usp=sharing}{CODE HERE}
\\
\texttt{
0.0 0.0 1.0 1 \\
1.0 1.0 -1.0 1 \\
2.0 1.0 2.0 1 \\
3.0 2.0 0.0 \\
}
\begin{center}
    \includegraphics[width=7cm]{set1/images/p.png}
\end{center}

\end{solution}

\begin{problem}[4]
  A dataset $S = \{(\mathbf{x}_1, y_1),\cdots,(\mathbf{x}_N, y_N)\} \subset \mathbb{R}^d \times \mathbb{R}$ is \emph{linearly separable} if there exists a perceptron that correctly classifies all data points in the set. In other words, there exists a hyperplane that separates positive data points and negative data points.

  In 2D space, what is the minimum size of a dataset that is not linearly separable, such that no three points are collinear? How about the minimum size of a dataset in 3D that is not linearly separable, such that no four points are coplanar? Please limit your explanation to a few lines - you should justify but not prove your answer.

  Finally, how does this generalize to N-dimension? More precisely, in N-dimensional space, what is the minimum size of a dataset that is not linearly separable, such that no $N$ points are on the same hyperplane? For the $N$-dimensional case, you may state your answer without proof or justification.
\end{problem}
\begin{solution}
In 2D, it is four points. Allow the points to be organized in a square with a +1, -1 pattern (going around the perimeter). It is impossible to classify those points with a single perceptron. \\

Similarly, this (degenerate) case exists in 3D. However, If we ignore it, then 5 points (additional one for the extra dimension) will be non-separable by a perceptron (plane). \\

For $N$-dimensions, it generalizes (the break point) to N + 2 points
\end{solution}

\begin{problem}[2]
  Run the visualization code in the Jupyter notebook section corresponding to question C (report your plots). Assume a dataset is \emph{not} linearly separable. Will the Perceptron Learning Algorithm ever converge? Why or why not?
\end{problem}
\begin{solution}
    No, becuase the algorithm converges if and only if the data is linearly separable.

    \begin{center}
        \includegraphics[width=7cm]{set1/images/bruh.png}
    \end{center}
\end{solution}

\begin{problem}[2]
How does the convergence behavior of the weight vector differ between the perceptron and SGD algorithms? Think of comparing, at a high level, their smoothness and whether they always converge (You don't need to implement any code for this problem.)
\end{problem}
\begin{solution}
SGD updates more "continuously" since we follow the gradient. In contrast, the perceptron is being "decided" by several points, resulting in more sporadic behavior. SGD will always converge as there will almost always be some "minimum" point. The perceptron, on the other hand, only converges when we have linearly separable data (almost never).
\end{solution}
\end{document}